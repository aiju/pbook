\title{Topics in Mathematics and Physics}
\author{Julius Schmidt}
\date{nvwh@cuvpbqr.qr (rot13 encoded)}
\begin{document}
\maketitle
The purpose of these ``lectures'' is to introduce you to concepts in mathematics and physics.
Unlike a traditional textbook, this text is presented as a series of exercises.
It should go without saying that it is essential that you attempt all of them (except the ones marked ``optional'').
I hope to provide sufficient hints to make them accessible to as broad a range of skills as possible.

Note that I am a physicist and so what I refer to as ``mathematics'' is somewhat different from what mathematicians consider it to be.
I will not hesitate to ignore issues of convergence and so forth in this text, as I believe them to be largely irrelevant to the underlying concepts involved.

This is a work in progress and topics will appear in no particular order.
\input{fourier}
\end{document}
