\documentclass{article}
\usepackage{amsmath}
\usepackage{amssymb}
\usepackage{amsfonts}
\usepackage{amsbsy}
\usepackage{mathrsfs}
\usepackage{tikz}
\usepackage{hyperref}
\usepackage[left=2cm,right=2cm,top=2cm,bottom=2cm]{geometry}
\DeclareFontFamily{U}{wncy}{}
\DeclareFontShape{U}{wncy}{m}{n}{<->wncyr10}{}
\DeclareSymbolFont{mcy}{U}{wncy}{m}{n}
\DeclareMathSymbol{\Sh}{\mathord}{mcy}{"58}

\newcommand\im{\mathrm{i}}
\newcommand\D[1]{\mathrm{d}{#1}}
\newcommand\pfrac[2]{\frac{\partial{#1}}{\partial{#2}}}
\newcommand\Dfrac[2]{\frac{\D{#1}}{\D{#2}}}
\setlength\parindent{0mm}
\setlength\parskip{12pt}
\newenvironment{question}{
\par
	\begin{tabular}{|p{15cm}}{\bf Question.}
}{
	\end{tabular}
}
\newenvironment{definition}{
\par
	\begin{tabular}{|p{15cm}}{\bf Definition.}
}{
	\end{tabular}
}
\newcommand\result[1]{\par{\bf Result.} #1\par}
\newcommand\hint[1]{\par{\bf Hint.} #1\par}
\newcommand\solution[1]{\par{\bf Solution.} #1\par}
\newcommand\theorem[1]{\par{\bf Theorem.} #1\par}

\newcommand\ket[1]{\left|{#1}\right\rangle}
\newcommand\bra[1]{\left\langle{#1}\right|}
\newcommand\braket[2]{\left\langle{#1}|{#2}\right\rangle}
\DeclareMathOperator{\tr}{tr}
\renewcommand\Re{\operatorname{Re}}
\renewcommand\Im{\operatorname{Im}}
\title{Topics in Mathematics and Physics}
\author{Julius Schmidt}
\date{nvwh@cuvpbqr.qr (rot13 encoded)}
\begin{document}
\maketitle
The purpose of these ``lectures'' is to introduce you to concepts in mathematics and physics.
Unlike a traditional textbook, this text is presented as a series of exercises.
It should go without saying that it is essential that you attempt all of them (except the ones marked ``optional'').
I hope to provide sufficient hints to make them accessible to as broad a range of skills as possible.

Note that I am a physicist and so what I refer to as ``mathematics'' is somewhat different from what mathematicians consider it to be.
The aim of this text is to develop an intuition for the concepts involved, as well as develop your skills in applying these concepts.
It is not meant to be a formal development and as such issues of convergence and the like will often be ignored.

This is a work in progress and topics will appear in no particular order.

The \LaTeX source code for this text is available at \url{https://github.com/aiju/pbook} under a MIT license.
\input{fourier}
\input{complana}
\end{document}

